
% symbols missing
% These are available in Unicode as ⟦ (U+27E6) and ⟧ (U+27E7),
% \newcommand{\llbracket}{\text{⟦}}
% \newcommand{\rrbracket}{\text{⟧}}
\newcommand{\llbracket}{[\![}
\newcommand{\rrbracket}{]\!]}

\newcommand{\mapsfrom}{\leftarrow\!\shortmid}
% \newcommand{\mapsto}{\!\shortmid\!}
\newcommand{\maps}{\mapsto}


\newcommand{\exc}{\vmath{\colF exec}} % Executation $\exc: \impsp \rightarrow \funsp$
\newcommand{\eval}{\vmath{\colR eval}} % Evaluation $\eval: \impsp \rightarrow \ressp$

\newcommand{\ufloor}{{\vmath{\colL floor}}}
\newcommand{\uceil}{{\vmath{\colU ceil}}}

\newcommand{\triv}{\text{Triv}}

\newcommand{\idFunc}{\mathrm{Id}}
\newcommand{\reals}{\mathbb{R}}

% fonts missing
\newcommand{\One}{\mathbf{1}} % \usepackage{bbm}

\renewcommand{\mathbbm}[1]{\mathbb{#1}}
\renewcommand{\mathscr}[1]{\mathcal{#1}}

\newcommand{\nonNegRealsComp}{\Rcomp}
\newcommand{\nonNegReals}{\mathbb{R}_+}
% symbols missing
% http://www-sop.inria.fr/marelle/tralics/quadrat/testmath.xml
\newcommand{\varocircle}{⦾}
\newcommand{\varotimes}{⊗}
\newcommand{\varovee}{(\vee)}

\newcommand{\vmath}[1]{\mathsf{#1}} % How words should appear in math mode.

\newcommand{\colR}{\color{darkred}}
\newcommand{\colF}{\color{darkgreen}}
\newcommand{\colH}{\color{blue}}

\newcommand{\rtof}{{\colH \varphi}}
\newcommand{\ftor}{{\colH h}}
\newcommand{\ftoR}{{\colH H}}
% \newcommand{\ftoR}{{\color{black}H}}
\newcommand{\Rcomp}{\overline{\mathbb{R}}_{+}}
\newcommand{\Rcpu}[1]{\Rcomp^{\lower{13.5pt}{\textrm{[#1]}}}}


% \newcommand{\colR}{\color[rgb]{0.555789,0.000000,0.000000}}
% \newcommand{\colF}{\color[rgb]{0.094869,0.500000,0.000000}}
% \newcommand{\colH}{\color[rgb]{0.000000,0.400000,1.000000}}
\newcommand{\colI}{\color[RGB]{214,120,5}}



% \newcommand{\posA}{\mathcal{P}}
% \newcommand{\posB}{\mathcal{Q}}

% \newcommand{\posAleq}{\posleq_{\posA}}
% \newcommand{\posBleq}{\posleq_{\posB}}
% \newcommand{\posleq}{\preceq}

% \newcommand{\antichains}{\mathrm{A}}
% \newcommand{\upsets}{\mathrm{U}}

% \newcommand{\CPO}{CPO}

% \newcommand{\Min}{\mathsf{Min}}
% \newcommand{\Max}{\mathsf{Max}}

% \newcommand{\pset}{\mathscr{P}}

% \newcommand{\upit}{\uparrow}

% \newcommand{\lfp}{\mathsf{lfp}}
% \newcommand{\Rcomp}{\overline{\mathbb{R}}_{+}}

% \newcommand{\reals}{\mathbb{R}}
% \newcommand{\nonNegReals}{\mathbb{R}_{+}}
% \newcommand{\nonNegRealsComp}{\Rcomp}


\newcommand{\funsp}{{\colF\mathscr{F}}} % Function space
\newcommand{\funleq}{\posleq_{\funsp}} % Function space
\newcommand{\fun}{\vmath{\colF f}} % Function
\newcommand{\res}{\vmath{\colR r}} % Resources
\newcommand{\funtop}{\top_\funsp}
\newcommand{\funbot}{\bot_\funsp}

\newcommand{\imp}{\vmath{i}} % Implementation
\newcommand{\impsp}{{\colI\mathscr{I}}} % Implementation space


% \newcommand{\param}{\vmath{p}} % Parameter
\newcommand{\paramsp}{\mathscr{P}} % Parameter space
% \newcommand{\paramleq}{\leq_{\paramsp}}


\newcommand{\resleq}{\posleq_{\ressp}}
\newcommand{\restop}{\top_\ressp}
\newcommand{\resbot}{\bot_\ressp}

\newcommand{\ressp}{{\colR\mathscr{R}}} % Resources space
\newcommand{\resspleq}{\resleq}

\newcommand{\tressp}{\trof(\ressp)} % Trade-off space
\newcommand{\trof}{\mathscr{T}} % Trade-off space
\newcommand{\tres}{T}
\newcommand{\tresleq}{\leq_{\trof}} % Trade-off space
\newcommand{\trleq}{\leq_{\trof}} % Trade-off space

%:section:articles/1509-gcmdp:



\newcommand{\dpisp}{\ensuremath{\vmath{DPI}}}
\newcommand{\cdpisp}{\ensuremath{\vmath{CDPI}}}

\newcommand{\dprobsp}{\ensuremath{\vmath{DP}}}
\newcommand{\dprob}{\vmath{dp}} % Design problem


\newcommand{\dpseries}{\vmath{series}}
\newcommand{\dppar}{\vmath{par}}
\newcommand{\dploop}{\vmath{loop}}

\newcommand{\dploopb}{\vmath{loopb}} % second form of dploop

\newcommand{\cdprobsp}{\ensuremath{\vmath{CDP}}}
\newcommand{\cdprob}{\vmath{cdp}} % Design problem
\newcommand{\dpatoms}{\vmath{atoms}} % Atoms of a cdp


\newcommand{\resMin}{{\Min_{\resleq}}}



\newcommand{\unconnectedfun}{\mathsf{UF}}
\newcommand{\unconnectedres}{\mathsf{UR}}



\newcommand{\Aressp}{{\colR\mathsf{\colR A}\ressp}} % Antichains of resources
\newcommand{\Afunsp}{{\colF\mathsf{\colF A}\funsp}} % Antichains of functions


%:section:articles/UDP:

\newcommand{\udpa}{\boldsymbol{u}_a}
\newcommand{\udpb}{\boldsymbol{u}_b}
\newcommand{\udpL}{\boldsymbol{\mathsf{L}}}
\newcommand{\udpU}{\boldsymbol{\mathsf{U}}}
\newcommand{\udpsp}{\vmath{UDP}}
\newcommand{\udpleq}{\posleq_\udpsp}

\newcommand{\dpsp}{\vmath{DP}}
\newcommand{\dpleq}{\posleq_\dpsp}

\newcommand{\terms}{\vmath{Terms}}

\newcommand{\udpsem}{\Phi}

\newcommand{\dpsem}{\varphi}

\newcommand{\atoms}{\mathcal{A}}
\newcommand{\atree}{\boldsymbol{\vmath{T}}}
\newcommand{\val}{\boldsymbol{v}} % Valuation

\newcommand{\ops}{\vmath{ops}} % Set of operations

\newcommand{\ftorL}{\ftor_L}
\newcommand{\ftorU}{\ftor_U}

\newcommand{\acprod}{\mathbin{\boldsymbol{\times}}} % Product of antichains

\newcommand{\oploop}{\dagger}
\newcommand{\opseries}{\mathbin{\varocircle}}
\newcommand{\oppar}{\mathbin{\varotimes}}
\newcommand{\opcoprod}{\mathbin{\varovee}}

\newcommand{\UId}{\vmath{UId}} % Uncertain identity
\newcommand{\vdc}{\vmath{vdc}} % Van Der Corput


\newcommand{\makedp}{\Gamma}

\newcommand{\colU}{\color{purple}}
\newcommand{\colL}{\color{orange}}




\newcommand{\R}[1]{{\colR #1}}
\newcommand{\F}[1]{{\colF #1}}
\newcommand{\I}[1]{{\colI #1}}


\newcommand{\cdpiN}{\mathcal{V}} % Nodes in a CDPI
\newcommand{\cdpin}{v} % one node in a CDPI
\newcommand{\cdpinA}{v_1}
\newcommand{\cdpinB}{v_2}
\newcommand{\cdpiresind}{i}
\newcommand{\cdpifunind}{j}
\newcommand{\cdpiresindA}{i_1}
\newcommand{\cdpifunindB}{j_2}
\newcommand{\dpinumf}{\vmath{nf}}
\newcommand{\dpinumr}{\vmath{nr}}
\newcommand{\cdpinnumf}{\dpinumf_\cdpin} % Number of functionalities
\newcommand{\cdpinnumr}{\dpinumr_\cdpin} % Number of resources
\newcommand{\cdpiE}{\mathcal{E}} % Edges in a CDPI





% \let\oldfun\fun     \renewcommand{\fun}{{\colF\oldfun}}
% \let\oldres\res     \renewcommand{\res}{{\colR\oldres}}
% \let\oldimp\imp     \renewcommand{\imp}{{\colI\oldimp}}

% \let\oldexc\exc     \renewcommand{\exc}{{\colF\oldexc}}
% \let\oldeval\eval     \renewcommand{\eval}{{\colR\oldeval}}

% \let\oldfunsp\funsp \renewcommand{\funsp}{{\colF\oldfunsp}}
% \let\oldressp\ressp \renewcommand{\ressp}{{\colR\oldressp}}
% \let\oldimpsp\impsp \renewcommand{\impsp}{{\colI\oldimpsp}}

% \let\oldftor\ftor \renewcommand{\ftor}{{\colH\oldftor}}
% \let\oldrtof\rtof \renewcommand{\rtof}{{\colH\oldrtof}}

% % TODO: move away

% \renewcommand{\Aressp}{{\colR\antichains\ressp}} % Antichains of resources
% \renewcommand{\Afunsp}{{\colF\antichains\funsp}} % Antichains of resources






%
%:section:basic/optimization: Optimization staff


\newcommand{\subto}{\text{s.t.}} % Subject to in math
\newcommand{\with}{\text{using}} % "With"

%
%:section:basic/posets: Partial orders
%

\newcommand{\pset}{\mathscr{P}} % Power set (latenative to powerset
\DeclareMathOperator*{\Min}{Min}
\DeclareMathOperator*{\Inf}{Inf}
\DeclareMathOperator*{\Sup}{Sup}
\DeclareMathOperator*{\Max}{Max}

\newcommand{\lowerbounds}{\vmath{lowerbounds}}
\newcommand{\upperbounds}{\vmath{upperbounds}}
\newcommand{\posMin}{\Min}
\newcommand{\posleq}{\preceq}
% \newcommand{\poslneq}{\precneq} % usepackage{mathabx}
\newcommand{\poslt}{\prec}
\newcommand{\posgeq}{\succeq}

\newcommand{\posA}{\mathcal{P}} %

\newcommand{\posAleq}{\mathrel{{\posleq_\posA}}} %

\newcommand{\posAMin}{\mathop{{\posMin_{\posAleq}}}} %  Minimal elements

\newcommand{\posAmin}{\mathop{{\min_{\posAleq}}}} %  The least element
\newcommand{\posAmax}{\mathop{{\max_{\posAleq}}}} %  The least element

\newcommand{\posB}{\mathcal{Q}} %
\newcommand{\posBleq}{\mathrel{{\posleq_\posB}}} %
\newcommand{\posC}{\mathcal{R}} %


\newcommand{\lfp}{\vmath{lfp}} % Least fixed point
\newcommand{\prefixed}{\vmath{prefixed}} % prefixed points
\newcommand{\CPOs}{\textsc{CPO}s\xspace}
\newcommand{\CPO}{\textsc{CPO}\xspace}
\newcommand{\DCPOs}{\textsc{DCPO}s\xspace}
\newcommand{\DCPO}{\textsc{DCPO}\xspace}

%:example: The upper sets of P are $\upsets(P)$
% \newcommand{\antichains}{\vmath{antichains}} %
\newcommand{\antichains}{\vmath{A}} %
%:example: The antichains sets of P are $\antichains(P)$

\newcommand{\upsets}{\vmath{U}} %
%:example: The upper sets of $\posA$ are $\upsets\posA$
\newcommand{\downsets}{\vmath{D}} %
%:example: The down sets of $\posA$ are $\downsets\posA$


\newcommand{\upresleq}{\posleq_{\upressp}}
\newcommand{\upressp}{\upsets\ressp}
\newcommand{\allupsets}{\vmath{Up}} %
%:deprecated:


\newcommand{\upit}{{\uparrow\,}} % Converts to smallest upset containing the ste

\newcommand{\stupit}{\dot{\upit}} % Strict upper closure

\newcommand{\posetwidth}{\vmath{width}}
\newcommand{\posetheight}{\vmath{height}}

% reits
\newcommand{\posdef}[1]{\mathcal{P}_{#1}} % Positive definite matrices
\newcommand{\MR}{\M{R}}


%:section:0typography:Basic typography

\newcommand{\myacronym}[1]{\textsc{#1}\xspace} % All acronyms; good for text as well as math mode. Use lower case.

%:section:0typography/tensors:Tensors and tensor elements
\newcommand{\T}[1]{\boldsymbol{{\mathsf{#1}}}} % Tensor
%:nomenc:\T{A},\T{B},\T{C},\dots: Symbols denoting tensors
%:sort:---40a

\newcommand{\Tel}[1]{{\mathsf{#1}}} % Tensor element
%:nomenc:\Tel{A}_{ijk},\Tel{B}_{ijk},\Tel{C}_{ijk},\dots: Symbols denoting tensors elements
%:sort:---40b

\newcommand{\Te}[1]{\Tel{#1}}
%:deprecated:


%:section:0typography/matrices:Matrices and matrix elements
\newcommand{\M}[1]{\mathbf{#1}} %  A matrix
%:nomenc:\M{A},\M{B},\M{C},\dots: Symbols denoting matrices
%:sort:---30a

\newcommand{\Mel}[1]{\mathrm{#1}} % The elements of a matrix
%:nomenc:\Mel{A},\Mel{B},\Mel{C},\dots: Symbols denoting matrix elements
%:sort:---30b

%:section:0typography/sets:Sets
\newcommand{\aset}[1]{\mathscr{#1}} % A set
%:nomenc:\setA,\setB,\setC,\dots,\setM,\setX,\setY,\setZ:Symbols denoting sets.
%:sort:---10

\newcommand{\agroup}[1]{\mathrm{#1}} % Fonts for a set which is a group.
%:example: A set $\aset{X}$, a group $\agroup{X}$, $\agroup{G}$, \dots
%:nomenc:\grG,\grH,\grK,\grN,\dots:Symbols denoting groups.
%:sort:---50

\newcommand{\aseq}[1]{\boldsymbol{#1}} % Formatting for sequences
%:nomenc:\sqa,\sqb,\sqc,\dots:Symbols used for denoting sequences.
%:sort:---20a

\newcommand{\aseqe}[1]{#1} % Formatting for one element in a sequence
%:nomenc:\sqae_k,\sqbe_k,\sqce_k,\dots:Symbols denoting the elements of sequences
%:sort:---20b

\newcommand{\dummyIndices}{}
%:nomenc:i,j,k,s,v,\dots:Symbols usually used as indices.


%:section:0typography/misc:Everything else
\newcommand{\aword}[1]{\mathsf{#1}} % How words should look like in formulas.
%:example: Consider the operator $\aword{scale}$, \dots

\newcommand{\vmath}[1]{\aword{#1}} % How words should appear in math mode.

\newcommand{\codefunc}[1]{\texttt{#1}\xspace} % Code functions
%:example: The function \codefunc{select}

\newcommand{\swpackage}[1]{\textsc{#1}\xspace} % Name of software packages
%:example: The package \swpackage{Procgraph}, \swpackage{ZMQ}, \swpackage{Unix} .


%:section:articles/optbody:Optimal design of body and mind


\newcommand{\MA}{\M{A}}
\newcommand{\MB}{\M{B}}
\newcommand{\MC}{\M{C}}
\newcommand{\MG}{\M{G}}
\newcommand{\MH}{\M{H}}

\newcommand{\ML}{\M{L}}
\newcommand{\MQ}{\M{Q}}
\newcommand{\MP}{\M{P}}
\newcommand{\MS}{\M{S}}
\newcommand{\MSigma}{\M{\Sigma}}
\newcommand{\MV}{\M{V}}
\newcommand{\MW}{\M{W}}


\newcommand{\SP}{P_{\text{s}}} % Sensing power
\newcommand{\AP}{P_{\text{a}}} % Actuation power
\newcommand{\SE}{E} % Stored energy
\newcommand{\ER}{r} % Trajectory efficiency ratio
\newcommand{\HP}{\Theta} % Heading precision
\newcommand{\np}{n} % Number of pixels
%:section:common/algebra: Algebra
\newcommand{\ones}{\boldsymbol{1}}
%:nomenc:\ones_n:A vector of all ones in $\reals^n$.
%:sort:1

\newcommand{\idMat}{\M{I}} % Identity matrix
%:nomenc:\idMat_n:Identity matrix of size $n\times n$.
%:sort:I

%%:def:def:idMat


\newcommand{\matTrace}{\vmath{Tr}} % Trace of a matrix.
%:sort:trace

%%:def:def:matrix-trace

\newcommand{\angleFun}{\angle} % Angle function
%:def:def:angle-function

\newcommand{\flatten}{\mathsf{vec}} % Matrix-to-vector rearrangement.
%:sort:vec

%%:def:def:vec




\newcommand{\batterymass}{{\colR m}}
\newcommand{\batterycapacity}{{\colF C}}
\newcommand{\batterycost}{{\colR c}}
\newcommand{\specificenergy}{{\colR \rho}}
\newcommand{\specificcost}{{\colR \alpha}}



\newcommand{\D}{\,\textrm{d}} % Used for integrals
%:nomenc:\D x:integration measure
%:nomenc-exclude:
%:sort:d

\newcommand{\ex}{\mathbb{E}}
